% !TeX root=../doc.tex
\section{Einleitung}

In dem Animationsfilm „World Invasion: Battle Informatikum“, welchen wir für das Projekt des Moduls „Interaktives Visuelles Computing“ (IVC) angefertigt haben, geht es um eine Armee von Außerirdischen die auf den Campus des Informatikums landen, mit der Absicht, diesen zu zerstören. Der Titel ist in Anlehnung an den US-amerikanischen Film „World Invasion: Battle Los Angeles“ aus dem Jahr 2011. Bei den Eindringlingen handelt es sich um die so genannten \textit{Minimons}, deren Name sich aus „mini-“ wie klein zusammensetzt und das „mon“ als Abkürzung für „Monster“ steht.

Was uns in erster Linie zu diesem Thema inspiriert hat ist ein Remix des Soundtracks zu dem im Jahre 2010 erschienenen Film ``Scott Pilgrim vs. the World'' von Robotaki (zu finden auf seinem \href{https://soundcloud.com/robotaki/scott-pilgrim-vs-the-world-fight-robotaki-remix}{Soundcloud}). Diesen Remix haben wir dann auch als Hintergrundmusik ausgewählt. Insbesondere das Abstimmen der Animation und der Handlung auf die Musik haben wir als eine äußerst interessante Challenge gesehen.

Der Film wurde mit \Povray für eine Bildrate von 24,5 Bildern pro Sekunde erzeugt. Er hat eine Gesamtspiellänge von 2 Minuten und 34 Sekunden. Die einzelnen Bilder wurden in dem für Kinoformate üblichen Seitenverhältnis 21:9 (2,33:1) und einer Höhenauflösung von 1080p aufgezeichnet. Dabei wurden insgesamt 3024 Bilder erzeugt die eine gesamte Dateigröße von 4,3 GB umfassten.
