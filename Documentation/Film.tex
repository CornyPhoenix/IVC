% !TeX root=../doc.tex
\subsection{Filmszenen}

Eine besondere Herausforderung stellte die Angleichung der Übergänge zwischen den Szenen an den Soundtrack dar, dessen Tempo 140 BPM beträgt. Wir haben uns dazu entschlossen eine Szene 32 Schläge (beziehungsweise 8 Viervierteltakte) lang zu machen. Das entspricht einer Dauer von 13,71 Sekunden für eine Szene, wie sich aus folgender Formel errechnet.

\begin{align*}
	\frac{60 \text{ s/min}}{140 \text{ Schläge/min}} \cdot 32 \text{ Schläge} = 13,71 \text{ s}
\end{align*}

Um die Anzahl der Bilder zu bestimmen, muss diese Dauer mit der Bildrate multipliziert werden. Als Bildrate haben wir 24,5 ausgesucht, um eine gerade Anzahl von Bilder pro Szene zu haben ($13,71 \text{ s} \cdot 24.5\text{ } \frac{\text{Bilder}}{\text{s}} = 336 \text{ Bilder}$).

Unter Berücksichtigung der Länge, die aus der Nachbearbeitung des Soundtracks resultiert ist, haben wir ungefähr 9 Szenen, in denen sich die gesamte Handlung abspielt. Eine genaue Übersicht über die Szenen sowie deren Startdauer und der Nummer des ersten Bildes, ist in der folgenden Tabelle gegeben.

\begin{figure}[h]
	\begin{tabular}{p{1cm}rrp{9cm}}
		\textbf{Szene} & \textbf{Zeit} & \textbf{Bildnr.} & \textbf{Beschreibung} \\
		\hline
		1 & 0,000 & & Weltallszene: Minimons fliegen auf die Erde zu\\
		2 & 28,622 & & Zoom an Informatikum\\
		3 & 42,111 & 0 & Minimon auf Haus D\\
		4 & 55,825 & 336 & Flug über Campus, Sprung von Haus D\\
		5 & 69,539 & 672 & Minimon landet und Armee tanzt\\
		6 & 83,254 & 1008 & Armee marschiert\\
		7 & 96,968 & 1344 & Verfolgung in Haus D, Schwenk\\
		8 & 110,682 & 1680 & Kamerafahrten über Häuser\\
		9 & 123,804 & 2001 & Explosion des Informatikums\\
		10 & 129,810 & 2352 & Abspann\\
	\end{tabular}
	\caption{Szenen des Films.}
	\label{fig:scenes}
\end{figure}
