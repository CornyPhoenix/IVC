% !TeX root=../doc.tex
\section{Weitere Werkzeuge}
	
	\subsection{\Illustrator}
	
	„\Illustrator [...] ist ein vektorbasiertes Grafik- und Zeichenprogramm. [...] Das Programm wurde 1987 von dem kalifornischen Softwareunternehmen Adobe Systems für den Apple \fremdw{Macintosh} entwickelt und wird bis heute von Adobe gepflegt und vermarktet. Es läuft auf den Computerbetriebssystemen Microsoft Windows sowie Apple Mac OS und gilt seit Jahren als Standardanwendung auf dem Gebiet der Vektorgrafik.“ \cite{Wiki-Illustrator}
	
	\subsection{Apple \iMovie}
	
	„\iMovie ist ein nicht-lineares Videoschnittprogramm für die Betriebssysteme OS X und iOS der Firma Apple.“ \cite{Wiki-iMovie}
	
	\subsection{\OSM}
	
	„\OSM ist ein freies Projekt, das für jeden frei nutzbare Geodaten sammelt (\fremdw{Open Data}). Mit Hilfe dieser Daten können Weltkarten errechnet oder Spezialkarten abgeleitet werden sowie Navigation betrieben werden. Auf der \OSM-Startseite ist eine solche Karte abrufbar.“ \cite{Wiki-OSM}
	
	\subsection{Google Earth}
	
	%TODO
	Für den Zoom auf das Informatikum aus dem Weltall.
	
	\subsection{\Arbaro}
	
	\Arbaro ist eine Implementierung des Baum-Generierungs-Algorithmus, welcher in \cite{Weber+95} beschrieben wurde. Es ist eine in Java geschriebene Open-Source-Software, mit dessen Hilfe für \Povray Koordinaten für die Umrisse eines Baums generiert werden können. Es ist bei \textit{Sourceforge} unter \url{http://sourceforge.net/projects/arbaro/} zu finden.